\documentclass[a4paper, 12pt]{article}
\usepackage{cmap}
\usepackage[12pt]{extsizes}			
\usepackage{mathtext} 				
\usepackage[T2A]{fontenc}			
\usepackage[utf8]{inputenc}			
\usepackage[english,russian]{babel}
\usepackage{setspace}
\singlespacing
\usepackage{amsmath,amsfonts,amssymb,amsthm,mathtools}
\usepackage{fancyhdr}
\usepackage{soulutf8}
\usepackage{euscript}
\usepackage{mathrsfs}
\usepackage{listings}

\usepackage[colorlinks=true, urlcolor=blue, linkcolor=black]{hyperref}

\pagestyle{fancy}
\usepackage{indentfirst}
\usepackage[top=10mm]{geometry}
\rhead{}
\lhead{}
\renewcommand{\headrulewidth}{0mm}
\usepackage{tocloft}
\renewcommand{\cftsecleader}{\cftdotfill{\cftdotsep}}
\usepackage[dvipsnames]{xcolor}

\lstdefinestyle{mystyle}{ 
	keywordstyle=\color{OliveGreen},
	numberstyle=\tiny\color{Gray},
	stringstyle=\color{BurntOrange},
	basicstyle=\ttfamily\footnotesize,
	breakatwhitespace=false,         
	breaklines=true,                 
	captionpos=b,                    
	keepspaces=true,                 
	numbers=left,                    
	numbersep=5pt,                  
	showspaces=false,                
	showstringspaces=false,
	showtabs=false,                  
	tabsize=2
}

\lstset{style=mystyle}

\begin{document}
\thispagestyle{empty}	
\begin{center}
	Московский авиационный институт
	
	(Национальный исследовательский университет)
	
	Факультет "Информационные технологии и прикладная математика"
	
	Кафедра "Вычислительная математика и программирование"
	
\end{center}
\vspace{40ex}
\begin{center}
	\textbf{\large{Лабораторная работа №5-7 по курсу\linebreak \textquotedblleft Операционные системы\textquotedblright}}
\end{center}
\vspace{35ex}
\begin{flushright}
	\textit{Студент: } Кочкожаров Иван Вячеславович
	
	\vspace{2ex}
	\textit{Группа: } М8О-208Б-22
	
	\vspace{2ex}
	\textit{Преподаватель: } Миронов Евгений Сергеевич
	
	\vspace{2ex}
	\textit{Вариант: } 17
	
	\vspace{2ex}
	\textit{Оценка: } \underline{\quad\quad\quad\quad\quad\quad}
	
	 \vspace{2ex}
	\textit{Дата: } \underline{\quad\quad\quad\quad\quad\quad}
	
	\vspace{2ex}
	\textit{Подпись: } \underline{\quad\quad\quad\quad\quad\quad}
	
\end{flushright}

\vspace{5ex}

\begin{vfill}
	\begin{center}
		Москва, 2023
	\end{center}	
\end{vfill}
\newpage

\begingroup
\color{black}
\tableofcontents\newpage
\endgroup

\section{Репозиторий}
\href{https://github.com/kochkozharov/os-labs}{https://github.com/kochkozharov/os-labs}

\section{Цель работы}
Приобретение практических навыков в:
\begin{itemize}
  \item Управлении серверами сообщений
  \item Применении отложенных вычислений
  \item Интеграции программных систем друг с другом
\end{itemize}

\section{Задание}
Реализовать распределенную систему по асинхронной обработке запросов. В данной распределенной системе должно существовать 2 вида узлов: «управляющий» и «вычислительный». Необходимо объединить данные узлы в соответствии с той топологией, которая определена вариантом. Связь между узлами необходимо осуществить при помощи технологии очередей сообщений. Также в данной системе необходимо предусмотреть проверку доступности узлов в соответствии с вариантом.

\section{Описание работы программы}
Топология 1:

Все вычислительные узлы находятся в списке. Есть только один управляющий узел. Чтобы
добавить новый вычислительный узел к управляющему, то необходимо выполнить команду:
create id -1.

Набор команд 4:

Поиск подстроки в строке. 

Формат команды: exec id text\_string pattern\_string.
result – номера позиций, где найден образец, разделенный точкой с запятой

Команда проверки 2:

Формат команды: ping id Команда проверяет доступность конкретного узла. Если узла нет, то необходимо выводить ошибку: «Error: Not found».

В ходе выполнения лабораторной работы использована библиотека ZeroMQ и следующие команды:
\begin{itemize}
  \item bind() - устанавливает "сокет" на адрес, а затем принимает входящие соединения на этом адресе.
  \item unbind() - отвязывает сокет от адреса
  \item connect() - создание соединения между сокетом и адресом
  \item disconnect() - разрывает соединение между сокетом и адресом
  \item send() - отправка сообщений
  \item recv() - получение сообщений
\end{itemize}


\newpage

\section{Исходный код}
socket.cpp
\begin{lstlisting}[language=C++]
#include <socket.h>

#include <iostream>
#include <map>
#include <optional>
#include <sstream>
#include <stdexcept>
#include <string>

static std::string GetAddress(int sockId) {
    constexpr int MAIN_PORT = 4000;
    return "tcp://127.0.0.1:" + std::to_string(MAIN_PORT + sockId);
}

Socket::Socket(zmq::socket_type t) : sock(ctx, t) {}

bool Socket::connect(int id) {
    try {
        sock.connect(GetAddress(id));
    } catch (...) {
        return false;
    }
    return true;
}

void Socket::disconnect(int id) { sock.disconnect(GetAddress(id)); }

void Socket::bind(int id) {
    try {
        sock.bind(GetAddress(id));
    } catch (...) {
        throw std::runtime_error("Error: port already in use");
    }
}

void Socket::unbind(int id) { sock.unbind(GetAddress(id)); }

bool Socket::sendMessage(const std::string &msg) {
    return sock.send(zmq::buffer(msg), zmq::send_flags::none).has_value();
}
std::optional<std::string> Socket::receiveMessage(bool nowait) {
    zmq::message_t zmsg{};
    auto len = sock.recv(
        zmsg, nowait ? zmq::recv_flags::dontwait : zmq::recv_flags::none);
    if (len) {
        return zmsg.to_string();
    }
    return {};
}
\end{lstlisting}

node.cpp
\begin{lstlisting}[language=C++]
#include "node.h"

#include "stdexcept"

ControlNode::ControlNode() : sock(zmq::socket_type::req) {}

ControlNode &ControlNode::get() {
    static ControlNode instance;
    return instance;
}

bool ControlNode::send(int id, const std::string &msg) {
    auto status = sock.connect(id) && sock.sendMessage(msg);
    return status;
}

std::optional<std::string> ControlNode::receive() {
    return sock.receiveMessage(false);
}

ComputationNode::ComputationNode(int id) : sock(zmq::socket_type::rep), id(id) {
    sock.bind(id);
}

ComputationNode::~ComputationNode() { sock.unbind(id); }

std::string ComputationNode::findAllOccurencies(const std::string &hay,
                                                const std::string &needle) {
    std::string repMsg = "";
    std::size_t pos = hay.find(needle, 0);
    while (pos != std::string::npos) {
        repMsg += std::to_string(pos) + ';';
        pos = hay.find(needle, pos + 1);
    }
    repMsg.pop_back();
    return repMsg;
}

void ComputationNode::computationLoop() {
    while (true) {
        auto reqMsg = sock.receiveMessage(false);
        std::stringstream ss(reqMsg.value());
        std::string command;
        ss >> command;
        if (command == "exec") {
            std::string hay, needle;
            ss >> hay >> needle;
            sock.sendMessage("Ok: " + std::to_string(id) + ' ' + findAllOccurencies(hay, needle));
        } else if (command == "ping") {
            sock.sendMessage("pong");
        }
    }
}

\end{lstlisting}

topology.cpp
\begin{lstlisting}[language=C++]
#include "topology.h"

Topology::NodeId::operator int() {
    return id;
}

Topology::TopoIter Topology::begin() {
    return {lists.begin(), lists.begin()->begin()};
}
Topology::TopoIter Topology::end() { return {lists.end(), lists.end()->end()}; }

bool Topology::insert(NodeId id, NodeId parentId) {
    auto coord = find(id);
    if (coord != end() || id == -1) {
        std::cerr << "Error: Already exists\n";
        return false;
    }
    if (parentId == -1) {
        lists.insert(lists.end(), std::list(1, id));
        return true;
    }
    auto coordParent = find(parentId);
    if (coordParent == end()) {
        std::cerr << "Error: Parent not found\n";
        return false;
    }
    if (coordParent.it2 != (coordParent.it1)->end()) {
        coordParent.it2++;
    }
    coordParent.it1->insert(coordParent.it2, id);
    return true;
}

Topology::TopoIter Topology::find(NodeId id) {
    auto it = begin();
    for (size_t i = 0; i < lists.size(); ++i) {
        it.it2 = it.it1->begin();
        for (size_t j = 0; j < it.it1->size(); ++j) {
            if (*it.it2 == id) {
                return it;
            }
            it.it2++;
        }
        it.it1++;
    }
    return end();
}

bool Topology::erase(NodeId id) {
    auto coord = find(id);
    if (coord == end() || id == -1) {
        return false;
    }
    coord.it1->erase(coord.it2);
    if (coord.it1->size() == 0) {
        lists.erase(coord.it1);
    }
    return true;
}

bool operator==(Topology::TopoIter it1, Topology::TopoIter it2) {
    return it1.it1 == it2.it1 && it1.it2 == it2.it2;
}

bool operator!=(Topology::TopoIter it1, Topology::TopoIter it2) {
    return !(it1 == it2);
}

bool Topology::contains(NodeId id) { return find(id) != end(); }

\end{lstlisting}

client.cpp
\begin{lstlisting}[language=C++]
#include <unistd.h>

#include <iostream>

#include "node.h"
#include "topology.h"
 #include <signal.h>

int main(int argc, char *argv[]) {
    if (argc != 2) {
        std::cerr << "Not enough arguments\n";
        std::exit(EXIT_FAILURE);
    }
    std::string command;
    std::cout << "> ";

    Topology topo;

    while (std::cin >> command) {
        if (command == "create") {
            int id, parentId;
            std::cin >> id >> parentId;

            pid_t pid = fork();
            if (pid == -1) {
                std::perror("fork");
                std::exit(EXIT_FAILURE);
            }
            if (pid == 0) {
                execl(argv[1], argv[1], std::to_string(id).c_str(),
                      std::to_string(parentId).c_str(), nullptr);

            } else {
                if (!topo.insert(Topology::NodeId(id, pid), parentId)) {
                    std::cout << "> ";
                    std::cout.flush();
                    continue;
                }
                std::cout << "Ok: " + std::to_string(pid) + '\n';
            }
        } else if (command == "ping") {
            int id;
            std::cin >> id;
            if (!topo.contains(id)) {
                std::cerr << "Error: Not found\n";
                std::cout << "> ";
                std::cout.flush();
                continue;
            }
            if (!ControlNode::get().send(id, "ping")) {
                std::cout << "Ok: 0\n";
                std::cout << "> ";
                std::cout.flush();
                continue;
            }
            auto response = ControlNode::get().receive();
            if (response == "pong") {
                std::cerr << "Ok: 1\n";
            } else {
                std::cerr << "Ok: 0\n";
            }
        } else if (command == "exec") {
            int id;

            std::string hay, needle;
            std::cin >> id >> hay >> needle;
            if (!topo.contains(id)) {
                std::cerr << "Error: " + std::to_string(id) + " Not found\n";
                std::cout << "> ";
                std::cout.flush();
                continue;
            }
            if (!ControlNode::get().send(id, "exec " + hay + ' ' + needle)) {
                std::cerr << "Error: Node is unavailable";
                std::cout << "> ";
                std::cout.flush();
                continue;
            }
            auto response = ControlNode::get().receive();
            if (response) {
                std::cout << *response << '\n';
            }
        } else {
            std::cout << "Unknown command\n";
        }

        std::cout << "> ";
        std::cout.flush();
    }
    auto it = topo.begin();
    auto end = topo.end();
    for (;it.it1 != end.it1; it.it1++) {
        it.it2 = it.it1->begin();
        for (;it.it2 != it.it1->end(); it.it2++) {
            kill(it.it2->pid, SIGKILL);
            it.it2++;
        }
        it.it1++;
    }
}
\end{lstlisting}

server.cpp
\begin{lstlisting}[language=C++]
#include <node.h>
#include <iostream>


int main(int argc, char *argv[]) {
    if (argc != 3) {
        std::cerr << "Not enough arguments\n";
        std::exit(EXIT_FAILURE);
    }
    int id = std::stoi(argv[1]);
    ComputationNode cn(id);
    cn.computationLoop();
    return 0;
}
\end{lstlisting}

\newpage
\section{Тесты}

lab5\_test.cpp
\begin{lstlisting}[language=C++]
#include <gtest/gtest.h>
#include <signal.h>

#include "node.h"
#include "topology.h"

TEST(Lab5Tests, TopologyF) {
    Topology t;
    ASSERT_EQ(t.find(100), t.end());
    ASSERT_TRUE(t.insert(1, -1));
    ASSERT_EQ(t.find(1), t.begin());
    ASSERT_TRUE(t.insert(2, 1));
    auto it = t.begin();
    it.it2++;
    ASSERT_EQ(t.find(2), it);
    ASSERT_TRUE(t.insert(66, 2));
    it.it2++;
    ASSERT_EQ(t.find(66), it);
    ASSERT_TRUE(t.insert(3, -1));
    it.it1++;
    it.it2 = it.it1->begin();
    ASSERT_EQ(t.find(3), it);
    ASSERT_TRUE(t.insert(7, 3));
    it.it2++;
    ASSERT_EQ(t.find(7), it);
    ASSERT_FALSE(t.insert(2, -1));
    ASSERT_FALSE(t.insert(6, -4));
    ASSERT_EQ(t.find(100), t.end());
    ASSERT_TRUE(t.erase(1));
    ASSERT_EQ(t.find(1), t.end());
    ASSERT_EQ(t.find(2), t.begin());
    ASSERT_TRUE(t.erase(2));
    ASSERT_TRUE(t.erase(66));
    ASSERT_EQ(t.find(3), t.begin());
    ASSERT_TRUE(t.insert(2, 3));
    it = t.begin();
    it.it2++;
    ASSERT_EQ(t.find(2), it);
}

TEST(Lab5Tests, CalculationTest) {
    ASSERT_EQ(ComputationNode::findAllOccurencies("memmem", "mem"), "0;3");
    ASSERT_EQ(ComputationNode::findAllOccurencies("1000", "00"), "1;2");
}

TEST(Lab5Tests, ExecTest) {
    auto cstr = std::getenv("PATH_TO_SERVER");
    ASSERT_TRUE(cstr);
    pid_t pid = fork();
    if (pid == -1) {
        std::perror("fork");
        std::exit(EXIT_FAILURE);
    }
    if (pid == 0) {
        execl(cstr, cstr, std::to_string(1).c_str(), std::to_string(-1).c_str(),
              nullptr);
    }
    ControlNode::get().send(1, "exec ooloo oo");
    ASSERT_EQ(ControlNode::get().receive().value(), "Ok: 1 0;3");
    kill(pid, SIGKILL);
}
\end{lstlisting}

\newpage

\section{Консоль}
\begin{verbatim}
ivan@asus-vivobook ~/c/o/b/lab5_7 (reports) > client ./server
> create 1 -1
Ok: 78019
> create 2 1
Ok: 78038
> create 3 -1
Ok: 78060
> exec 3 memmem mem
Ok: 3 0;3
> ping 3
Ok: 1
> ping 4
Error: Not found
> exec 1 2222 2 
Ok: 1 0;1;2;3
> exit
\end{verbatim}

\section{Запуск тестов}
\begin{verbatim}
ivan@asus-vivobook ~/c/o/b/tests (reports) [SIGINT]> PATH_TO_SERVER=../lab5_7/server lab5_7_test
Running main() from /var/tmp/portage/dev-cpp/gtest-1.13.0/work/googletest-1.13.0/googletest/src/gtest_main.cc
[==========] Running 3 tests from 1 test suite.
[----------] Global test environment set-up.
[----------] 3 tests from Lab5Tests
[ RUN      ] Lab5Tests.TopologyF
Error: Already exists
Error: Parent not found
[       OK ] Lab5Tests.TopologyF (0 ms)
[ RUN      ] Lab5Tests.CalculationTest
[       OK ] Lab5Tests.CalculationTest (0 ms)
[ RUN      ] Lab5Tests.ExecTest
[       OK ] Lab5Tests.ExecTest (127 ms)
[----------] 3 tests from Lab5Tests (128 ms total)

[----------] Global test environment tear-down
[==========] 3 tests from 1 test suite ran. (128 ms total)
[  PASSED  ] 3 tests.
\end{verbatim}
\newpage

\section{Выводы}

В результате выполнения данной лабораторной работы была реализована распределенная система по асинхронной обработке запросов в соответствие с вариантом задания на С++. Приобретены практические навыки в управлении серверами сообщений, применении отложенных вычислений и интеграции программных систем друг с другом.
\end{document}