\documentclass[a4paper, 12pt]{article}
\usepackage{cmap}
\usepackage[12pt]{extsizes}			
\usepackage{mathtext} 				
\usepackage[T2A]{fontenc}			
\usepackage[utf8]{inputenc}			
\usepackage[english,russian]{babel}
\usepackage{setspace}
\singlespacing
\usepackage{amsmath,amsfonts,amssymb,amsthm,mathtools}
\usepackage{fancyhdr}
\usepackage{soulutf8}
\usepackage{euscript}
\usepackage{mathrsfs}
\usepackage{listings}

\usepackage[colorlinks=true, urlcolor=blue, linkcolor=black]{hyperref}

\pagestyle{fancy}
\usepackage{indentfirst}
\usepackage[top=10mm]{geometry}
\rhead{}
\lhead{}
\renewcommand{\headrulewidth}{0mm}
\usepackage{tocloft}
\renewcommand{\cftsecleader}{\cftdotfill{\cftdotsep}}
\usepackage[dvipsnames]{xcolor}

\lstdefinestyle{mystyle}{ 
	keywordstyle=\color{OliveGreen},
	numberstyle=\tiny\color{Gray},
	stringstyle=\color{BurntOrange},
	basicstyle=\ttfamily\footnotesize,
	breakatwhitespace=false,         
	breaklines=true,                 
	captionpos=b,                    
	keepspaces=true,                 
	numbers=left,                    
	numbersep=5pt,                  
	showspaces=false,                
	showstringspaces=false,
	showtabs=false,                  
	tabsize=2
}

\lstset{style=mystyle}

\begin{document}
\thispagestyle{empty}	
\begin{center}
	Московский авиационный институт
	
	(Национальный исследовательский университет)
	
	Факультет "Информационные технологии и прикладная математика"
	
	Кафедра "Вычислительная математика и программирование"
	
\end{center}
\vspace{40ex}
\begin{center}
	\textbf{\large{Лабораторная работа №1 по курсу\linebreak \textquotedblleft Операционные системы\textquotedblright}}
\end{center}
\vspace{35ex}
\begin{flushright}
	\textit{Студент: } Кочкожаров Иван Вячеславович
	
	\vspace{2ex}
	\textit{Группа: } М8О-208Б-22
	
	\vspace{2ex}
	\textit{Преподаватель: } Миронов Евгений Сергеевич
	
	\vspace{2ex}
	\textit{Вариант: } 20 
	
	\vspace{2ex}
	\textit{Оценка: } \underline{\quad\quad\quad\quad\quad\quad}
	
	 \vspace{2ex}
	\textit{Дата: } \underline{\quad\quad\quad\quad\quad\quad}
	
	\vspace{2ex}
	\textit{Подпись: } \underline{\quad\quad\quad\quad\quad\quad}
	
\end{flushright}

\vspace{5ex}

\begin{vfill}
	\begin{center}
		Москва, 2023
	\end{center}	
\end{vfill}
\newpage


\begingroup
\color{black}
\tableofcontents\newpage
\endgroup

\section{Репозиторий}
\href{https://github.com/kochkozharov/os-labs}{https://github.com/kochkozharov/os-labs}

\section{Цель работы}
Приобретение практических навыков в:
\begin{itemize}
  \item Управлении процессами в ОС
  \item Обеспечении обмена данных между процессами посредством каналов
\end{itemize}

\section{Задание}
Составить и отладить программу на языке Си, осуществляющую работу с процессами и 
взаимодействие между ними в одной из двух операционных систем. В результате работы 
программа (основной процесс) должен создать для решение задачи один или несколько 
дочерних процессов. Взаимодействие между процессами осуществляется через системные 
сигналы/события и/или каналы (pipe).
Необходимо обрабатывать системные ошибки, которые могут возникнуть в результате работы.

\section{Описание работы программы}
Родительский процесс создает дочерний процесс. Первой строкой пользователь в консоль родительского процесса вводит имя файла, которое будет использовано для открытия File с таким 
именем на запись. Родительский и дочерний процесс представлены разными программами. Родительский процесс принимает от пользователя строки произвольной длины и пересылает их в 
pipe1. Процесс child проверяет строки на валидность правилу. Если строка соответствует правилу, то она выводится в стандартный поток вывода дочернего процесса, иначе в pipe2 выводится информация об ошибке. Родительский процесс полученные от child ошибки выводит в стандартный поток вывода.

В ходе выполнения лабораторной работы я использовала следующие системные вызовы:
\begin{itemize}
  \item fork() - создание нового процесса
  \item pipe() - создание канала
  \item dup2() - создание копии файлового дескриптора, используя для нового дескриптора самый маленький свободный номер файлового дескриптора.
  \item execlp() - запуск файла на исполнение
\end{itemize}


\newpage

\section{Исходный код}
lab1.c
\begin{lstlisting}[language=C++]
    #include "lab1.h"

    #include <errno.h>
    #include <fcntl.h>
    #include <stdlib.h>
    #include <sys/wait.h>
    #include <unistd.h>
    
    #include "error_handling.h"
    #include "utils.h"
    
    int ParentRoutine(const char* pathToChild, FILE* stream) {
        int fds[2] = {-1, -1};
        int pipes[2][2] = {{-1, -1}, {-1, -1}};
    
        char* line = NULL;
        size_t len = 0;
        char* extraLine = NULL;
        size_t extraLen = 0;
        
        errno = 0;
        ssize_t nread = getline(&line, &len, stream);
        GOTO_IF(errno == ENOMEM, "getline" , err);
        if (nread == -1) {
            fprintf(stderr, "Input 2 filenames\n");
            goto err;
        }
        line[nread - 1] = '\0';
    
        errno = 0;
        nread = getline(&extraLine, &extraLen, stream);
        GOTO_IF(errno == ENOMEM, "getline" , err);
        if (nread == -1) {
            fprintf(stderr, "Input 2 filenames\n");
            goto err;
        }
        extraLine[nread - 1] = '\0';
        
        fds[0] = open(line, O_CREAT | O_WRONLY | O_TRUNC, MODE);
        GOTO_IF(fds[0] == -1, "open", err);
        fds[1] = open(extraLine, O_CREAT | O_WRONLY | O_TRUNC, MODE);
        GOTO_IF(fds[1] == -1, "open", err);
        free(extraLine);
        extraLine = NULL;
    
        GOTO_IF(pipe(pipes[0]), "pipe", err);
        GOTO_IF(pipe(pipes[1]), "pipe", err);
    
        pid_t pids[2];
        pids[0] = fork();
        GOTO_IF(pids[0] == -1, "fork", err);
        if (pids[0]) {
            pids[1] = fork();
            GOTO_IF(pids[1] == -1, "fork", err);
        }
    
        if (pids[0] == 0) {  // child1
            ABORT_IF(close(fds[1]), "close");
            fds[1] = -1;
            ABORT_IF(close(pipes[1][READ_END]), "close");
            pipes[1][READ_END] = -1;
            ABORT_IF(close(pipes[1][WRITE_END]), "close");
            pipes[1][WRITE_END] = -1;
            ABORT_IF(close(pipes[0][WRITE_END]), "close");
            pipes[0][WRITE_END] = -1;
    
            GOTO_IF(dup2(pipes[0][READ_END], STDIN_FILENO) == -1, "dup2", err);
            ABORT_IF(close(pipes[0][READ_END]), "close");
            pipes[0][READ_END] = -1;
    
            GOTO_IF(dup2(fds[0], STDOUT_FILENO) == -1, "dup2", err);
            ABORT_IF(close(fds[0]), "close");
            fds[0] = -1;
    
            GOTO_IF(execl(pathToChild, "child", NULL), "execl", err);
        } else if (pids[1] == 0) {  // child2
            ABORT_IF(close(fds[0]), "close");
            fds[0] = -1;
            ABORT_IF(close(pipes[0][READ_END]), "close");
            pipes[0][READ_END] = -1;
            ABORT_IF(close(pipes[0][WRITE_END]), "close");
            pipes[0][WRITE_END] = -1;
            ABORT_IF(close(pipes[1][WRITE_END]), "close");
            pipes[1][WRITE_END] = -1;
    
            GOTO_IF(dup2(pipes[1][READ_END], STDIN_FILENO) == -1, "dup2", err);
            ABORT_IF(close(pipes[1][READ_END]), "close");
            pipes[1][READ_END] = -1;
    
            GOTO_IF(dup2(fds[1], STDOUT_FILENO) == -1, "dup2", err);
            ABORT_IF(close(fds[1]), "close");
            fds[1] = -1;
    
            GOTO_IF(execl(pathToChild, "child", NULL), "execl", err);
        } else {  // parent
            ABORT_IF(close(pipes[0][READ_END]), "close");
            pipes[0][READ_END] = -1;
            ABORT_IF(close(pipes[1][READ_END]), "close");
            pipes[1][READ_END] = -1;
            ABORT_IF(close(fds[0]), "close");
            fds[0] = -1;
            ABORT_IF(close(fds[1]), "close");
            fds[1] = -1;
    
            GOTO_IF(waitpid(-1, NULL, WNOHANG), "waitpid", err);
            ssize_t nread;
            while ((nread = getline(&line, &len, stream)) != -1) {
                GOTO_IF(waitpid(-1, NULL, WNOHANG), "waitpid", err);
                if (nread <= FILTER_LEN) {
                    GOTO_IF(write(pipes[0][WRITE_END], line, nread) == -1, "write", err);
                } else {
                    GOTO_IF(write(pipes[1][WRITE_END], line, nread) == -1, "write", err);
                }
            }
            GOTO_IF(errno == ENOMEM, "getline", err);
            ABORT_IF(close(pipes[0][WRITE_END]), "close");
            pipes[0][WRITE_END] = -1;
            ABORT_IF(close(pipes[1][WRITE_END]), "close");
            pipes[1][WRITE_END] = -1;
            for (int i = 0; i < 2; ++i) {
                int status;
                int waitPid = wait(&status);
                GOTO_IF(status || !(waitPid == pids[0] || waitPid == pids[1]), "wait", err);
            }
        }
    
        free(line);
        return 0;
    
    err:
        free(line);
        free(extraLine);
        errno = 0;
        close(fds[0]);
        close(fds[1]);
        close(pipes[0][READ_END]);
        close(pipes[0][WRITE_END]);
        close(pipes[1][READ_END]);
        close(pipes[1][WRITE_END]);
        if (errno == EIO) {
            abort();
        }
        return -1;
    }
\end{lstlisting}

utils.c
\begin{lstlisting}[language=C++]
#include "utils.h"

#include <stdio.h>
#include <stdlib.h>

void ReverseString(char* string, size_t length) {
    for (size_t i = 0; i < length >> 1; ++i) {
        char temp = string[i];
        string[i] = string[length - i - 1];
        string[length - i - 1] = temp;
    }
}
\end{lstlisting}

child.c
\begin{lstlisting}[language=C++]
#include <errno.h>
#include <stdlib.h>
#include <unistd.h>
#include <sys/wait.h>

#include "error_handling.h"
#include "utils.h"

int main(void) {
    ssize_t nread;
    char *line = NULL;
    size_t len;

    while ((nread = getline(&line, &len, stdin)) != -1) {
        ReverseString(line, nread-1);
        GOTO_IF(printf("%s", line) < 0, "printf", err);
    }
    GOTO_IF(errno == ENOMEM, "getline", err);

    free(line);
    return 0;

err:
    free(line);
    return -1;
}
\end{lstlisting}

main.c
\begin{lstlisting}[language=C++]
#include <stdio.h>
#include <stdlib.h>

#include "lab1.h"

int main(void) {
    const char *childPath = getenv("PATH_TO_CHILD");
    if (!childPath) {
        fprintf(stderr, "Set environment variable PATH_TO_CHILD\n");
        exit(EXIT_FAILURE);
    }
    return ParentRoutine(childPath , stdin);
}
\end{lstlisting}

\newpage
\section{Тесты}

\begin{lstlisting}[language=C++]
#include <gtest/gtest.h>

extern "C" {
#include "lab1.h"
}

#include <array>
#include <filesystem>
#include <fstream>
#include <memory>

namespace fs = std::filesystem;

TEST(FirstLabTests, SimpleTest) {
    const char *childPath = getenv("PATH_TO_CHILD");
    ASSERT_TRUE(childPath);

    const std::string fileWithInput = "input.txt";
    const std::string fileWithOutput1 = "output1.txt";
    const std::string fileWithOutput2 = "output2.txt";

    constexpr int outputSize1 = 5;
    constexpr int outputSize2 = 3;
    constexpr int inputSize = outputSize1 + outputSize2;

    const std::array<std::string, inputSize> input = {
        "rev",
        "revvvvv",
        "162te16782r18223r2",
        "",
        "r",
        "pqwrpqlwfpqwoglqwpglqpgwpqw.,g;q.wg;q.wg;w.qg;.w",
        "12345678900",
        "1234567890",
    };

    const std::array<std::string, outputSize1> expectedOutput1 = {
        "ver",
        "vvvvver",
        "",
        "r",
        "0987654321",
    };

    const std::array<std::string, outputSize2> expectedOutput2 = {
        "2r32281r28761et261",
        "w.;gq.w;gw.q;gw.q;g,.wqpwgpqlgpwqlgowqpfwlqprwqp",
        "00987654321",
    };

    {
        auto inFile = std::ofstream(fileWithInput);
        inFile << fileWithOutput1 << '\n'
               << fileWithOutput2 << '\n';
        for (const auto &line : input) {
            inFile << line << '\n';
        }
    }

    auto deleter = [](FILE *file) {
        fclose(file);
    };

    std::unique_ptr<FILE, decltype(deleter)> inFile(fopen(fileWithInput.c_str(), "r"), deleter);

    ASSERT_EQ(ParentRoutine(childPath, inFile.get()), 0);

    auto outFile1 = std::ifstream(fileWithOutput1);
    auto outFile2 = std::ifstream(fileWithOutput2);
    ASSERT_TRUE(outFile1.good() && outFile2.good());

    std::string result;

    for (const std::string &line : expectedOutput1) {
        std::getline(outFile1, result);
        EXPECT_EQ(result, line);
    }

    for (const std::string &line : expectedOutput2) {
        std::getline(outFile2, result);
        EXPECT_EQ(result, line);
    }

    auto removeIfExists = [](const std::string &path) {
        if (fs::exists(path)) {
            fs::remove(path);
        }
    };

    removeIfExists(fileWithInput);
    removeIfExists(fileWithOutput1);
    removeIfExists(fileWithOutput2);
}

TEST(FirstLabTests, ZeroOutputFileTest) {
    const char *childPath = getenv("PATH_TO_CHILD");
    ASSERT_TRUE(childPath);
    const std::string fileWithInput = "input.txt";

    {
        auto inFile = std::ofstream(fileWithInput);
    }
    auto deleter = [](FILE *file) {
        fclose(file);
    };

    std::unique_ptr<FILE, decltype(deleter)> inFile(fopen(fileWithInput.c_str(), "r"), deleter);

    ASSERT_EQ(ParentRoutine(childPath, inFile.get()), -1);
}

TEST(FirstLabTests, OneOutputFileTest) {
    const char *childPath = getenv("PATH_TO_CHILD");
    ASSERT_TRUE(childPath);
    const std::string fileWithInput = "input.txt";
    const std::string fileWithOutput = "output.txt";

    {
        auto inFile = std::ofstream(fileWithInput);
        inFile << fileWithOutput << '\n';
    }
    auto deleter = [](FILE *file) {
        fclose(file);
    };

    std::unique_ptr<FILE, decltype(deleter)> inFile(fopen(fileWithInput.c_str(), "r"), deleter);

    ASSERT_EQ(ParentRoutine(childPath, inFile.get()), -1);
    ASSERT_FALSE(fs::exists(fileWithOutput));
}

TEST(FirstLabTests, EmptyInputTest) {
    const char *childPath = getenv("PATH_TO_CHILD");
    ASSERT_TRUE(childPath);

    const std::string fileWithInput = "input.txt";
    const std::string fileWithOutput1 = "output1.txt";
    const std::string fileWithOutput2 = "output2.txt";

    {
        auto inFile = std::ofstream(fileWithInput);
        inFile << fileWithOutput1 << '\n'
               << fileWithOutput2 << '\n';
    }

    auto deleter = [](FILE *file) {
        fclose(file);
    };

    std::unique_ptr<FILE, decltype(deleter)> inFile(fopen(fileWithInput.c_str(), "r"), deleter);

    ASSERT_EQ(ParentRoutine(childPath, inFile.get()), 0);

    auto outFile1 = std::ifstream(fileWithOutput1);
    auto outFile2 = std::ifstream(fileWithOutput2);
    ASSERT_TRUE(outFile1.good() && outFile2.good());
    ASSERT_TRUE(fs::is_empty(fileWithOutput1) && fs::is_empty(fileWithOutput2));

    auto removeIfExists = [](const std::string &path) {
        if (fs::exists(path)) {
            fs::remove(path);
        }
    };

    removeIfExists(fileWithInput);
    removeIfExists(fileWithOutput1);
    removeIfExists(fileWithOutput2);
}
\end{lstlisting}

\newpage
\section{Демонстрация работы программы}

\begin{verbatim}
ivan@asus-vivobook ~/c/o/b/lab1 (reports)> PATH_TO_CHILD=child lab1
1
2
3l,34lgt,3,;34,;h534;h,43;lh543
ergerg,ererg
fd

d
gfd
gewegelh,;
sgl,r,ge;
fsdf
ivan@asus-vivobook ~/c/o/b/lab1 (reports)> ls
1  2  CMakeFiles/  CTestTestfile.cmake  Makefile  child*  cmake_install.cmake  lab1*
ivan@asus-vivobook ~/c/o/b/lab1 (reports)> cat 1
df

d
dfg
;,hlegeweg
;eg,r,lgs
fdsf
ivan@asus-vivobook ~/c/o/b/lab1 (reports)> cat 2
345hl;34,h;435h;,43;,3,tgl43,l3
grere,gregre
\end{verbatim}

\section{Запуск тестов}
\begin{verbatim}
ivan@asus-vivobook ~/c/o/build (reports) [1]> PATH_TO_CHILD=lab1/child tests/lab1_test
Running main() from /var/tmp/portage/dev-cpp/gtest-1.13.0/work/googletest-1.13.0/googletest/src/gtest_main.cc
[==========] Running 4 tests from 1 test suite.
[----------] Global test environment set-up.
[----------] 4 tests from FirstLabTests
[ RUN      ] FirstLabTests.SimpleTest
[       OK ] FirstLabTests.SimpleTest (6 ms)
[ RUN      ] FirstLabTests.ZeroOutputFileTest
Input 2 filenames
[       OK ] FirstLabTests.ZeroOutputFileTest (0 ms)
[ RUN      ] FirstLabTests.OneOutputFileTest
Input 2 filenames
[       OK ] FirstLabTests.OneOutputFileTest (0 ms)
[ RUN      ] FirstLabTests.EmptyInputTest
[       OK ] FirstLabTests.EmptyInputTest (4 ms)
[----------] 4 tests from FirstLabTests (10 ms total)
[----------] Global test environment tear-down
[==========] 4 tests from 1 test suite ran. (11 ms total)
[  PASSED  ] 4 tests.
\end{verbatim}
\newpage
\section{Выводы}

В результате выполнения данной лабораторной работы была написана программа на языке С++, осуществляющая работу с процессами и 
взаимодействие между ними. Преобретены практические навыки в управлении процессами в ОС и обеспечении обмена данных между процессами посредством каналов.
\end{document}