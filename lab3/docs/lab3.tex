\documentclass[a4paper, 12pt]{article}
\usepackage{cmap}
\usepackage[12pt]{extsizes}			
\usepackage{mathtext} 				
\usepackage[T2A]{fontenc}			
\usepackage[utf8]{inputenc}			
\usepackage[english,russian]{babel}
\usepackage{setspace}
\singlespacing
\usepackage{amsmath,amsfonts,amssymb,amsthm,mathtools}
\usepackage{fancyhdr}
\usepackage{soulutf8}
\usepackage{euscript}
\usepackage{mathrsfs}
\usepackage{listings}

\usepackage[colorlinks=true, urlcolor=blue, linkcolor=black]{hyperref}

\pagestyle{fancy}
\usepackage{indentfirst}
\usepackage[top=10mm]{geometry}
\rhead{}
\lhead{}
\renewcommand{\headrulewidth}{0mm}
\usepackage{tocloft}
\renewcommand{\cftsecleader}{\cftdotfill{\cftdotsep}}
\usepackage[dvipsnames]{xcolor}

\lstdefinestyle{mystyle}{ 
	keywordstyle=\color{OliveGreen},
	numberstyle=\tiny\color{Gray},
	stringstyle=\color{BurntOrange},
	basicstyle=\ttfamily\footnotesize,
	breakatwhitespace=false,         
	breaklines=true,                 
	captionpos=b,                    
	keepspaces=true,                 
	numbers=left,                    
	numbersep=5pt,                  
	showspaces=false,                
	showstringspaces=false,
	showtabs=false,                  
	tabsize=2
}

\lstset{style=mystyle}

\begin{document}
\thispagestyle{empty}	
\begin{center}
	Московский авиационный институт
	
	(Национальный исследовательский университет)
	
	Факультет "Информационные технологии и прикладная математика"
	
	Кафедра "Вычислительная математика и программирование"
	
\end{center}
\vspace{40ex}
\begin{center}
	\textbf{\large{Лабораторная работа №3 по курсу\linebreak \textquotedblleft Операционные системы\textquotedblright}}
\end{center}
\vspace{35ex}
\begin{flushright}
	\textit{Студент: } Кочкожаров Иван Вячеславович
	
	\vspace{2ex}
	\textit{Группа: } М8О-208Б-22
	
	\vspace{2ex}
	\textit{Преподаватель: } Миронов Евгений Сергеевич
	
	\vspace{2ex}
	\textit{Вариант: } 20 
	
	\vspace{2ex}
	\textit{Оценка: } \underline{\quad\quad\quad\quad\quad\quad}
	
	 \vspace{2ex}
	\textit{Дата: } \underline{\quad\quad\quad\quad\quad\quad}
	
	\vspace{2ex}
	\textit{Подпись: } \underline{\quad\quad\quad\quad\quad\quad}
	
\end{flushright}

\vspace{5ex}

\begin{vfill}
	\begin{center}
		Москва, 2023
	\end{center}	
\end{vfill}
\newpage

\begingroup
\color{black}
\tableofcontents\newpage
\endgroup

\section{Репозиторий}
\href{https://github.com/kochkozharov/os-labs}{https://github.com/kochkozharov/os-labs}

\section{Цель работы}
Приобретение практических навыков в:
\begin{itemize}
  \item Освоении принципов работы с файловыми системами
  \item Обеспечении обмена данных между процессами посредством технологии «File mapping»
\end{itemize}

\section{Задание}
Составить и отладить программу на языке Си, осуществляющую работу с процессами и взаимодействие между ними в одной из двух операционных систем. В результате работы программа (основной процесс) должен создать для решение задачи один или несколько дочерних процессов. Взаимодействие между процессами осуществляется через системные сигналы/события и/или через отображаемые файлы (memory-mapped files).
Необходимо обрабатывать системные ошибки, которые могут возникнуть в результате работы.

\section{Описание работы программы}
Задание аналогично первой лабораторной работе.

В ходе выполнения лабораторной работы я использовала следующие системные вызовы:
\begin{itemize}
  \item fork() - создание нового процесса
  \item sem\_open() - создание/открытие семафора
  \item sem\_post() - увеличивание значения семафора и разблокировка ожидающих потоков
  \item sem\_wait() - уменьшение значения семафора. Если 0, то вызывающий поток блокируется 
  \item sem\_close() - закрытие семафора
  \item shm\_open() - создание/открытие разделяемой памяти POSIX
  \item shm\_unlink() - закрытие разделяемой памяти
  \item ftruncate() - уменьшение длины файла до указанной  
  \item mmap() - отражение файла или устройства в памяти
  \item munmap() - снятие отражения
  \item execlp() - запуск файла на исполнение
  
\end{itemize}


\newpage

\section{Исходный код}
lab3.c
\begin{lstlisting}[language=C++]
#include "lab3.h"

#include <errno.h>
#include <fcntl.h>
#include <semaphore.h>
#include <signal.h>
#include <stdlib.h>
#include <sys/mman.h>
#include <sys/wait.h>
#include <time.h>
#include <unistd.h>

#include "error_handling.h"
#include "utils.h"

int ParentRoutine(const char* pathToChild, FILE* stream) {
    int fds[2] = {-1, -1};
    char* line = NULL;
    size_t len = 0;
    char* extraLine = NULL;
    size_t extraLen = 0;

    SharedMemory shm1;
    SharedMemory shm2;
    GOTO_IF(CreateSharedMemory(&shm1, SHARED_MEMORY_NAME_1, MAP_SIZE),
            "CreateSharedMemory", err);
    GOTO_IF(CreateSharedMemory(&shm2, SHARED_MEMORY_NAME_2, MAP_SIZE),
            "CreateSharedMemory", err);

    SemaphorePair pair1;
    SemaphorePair pair2;

    GOTO_IF(CreateSemaphorePair(&pair1, W_SEMAPHORE_NAME_1,
                                REV_SEMAPHORE_NAME_1) == -1,
            "CreateSemaphorePair", err);
    GOTO_IF(CreateSemaphorePair(&pair2, W_SEMAPHORE_NAME_2,
                                REV_SEMAPHORE_NAME_2) == -1,
            "CreateSemaphorePair", err);

    errno = 0;
    ssize_t nread = getline(&line, &len, stream);
    GOTO_IF(errno == ENOMEM, "getline", err);
    if (nread == -1) {
        fprintf(stderr, "Input 2 filenames\n");
        goto err;
    }
    line[nread - 1] = '\0';

    errno = 0;
    nread = getline(&extraLine, &extraLen, stream);
    GOTO_IF(errno == ENOMEM, "getline", err);
    if (nread == -1) {
        fprintf(stderr, "Input 2 filenames\n");
        goto err;
    }
    extraLine[nread - 1] = '\0';

    fds[0] = open(line, O_CREAT | O_WRONLY | O_TRUNC, MODE);
    GOTO_IF(fds[0] == -1, "open", err);
    fds[1] = open(extraLine, O_CREAT | O_WRONLY | O_TRUNC, MODE);
    GOTO_IF(fds[1] == -1, "open", err);
    free(extraLine);
    extraLine = NULL;

    pid_t pids[2];
    pids[0] = fork();
    GOTO_IF(pids[0] == -1, "fork", err);
    if (pids[0]) {
        pids[1] = fork();
        GOTO_IF(pids[1] == -1, "fork", err);
    }

    if (pids[0] == 0) {  // child1
        close(fds[1]);
        fds[1] = -1;

        GOTO_IF(dup2(shm1.fd, STDIN_FILENO) == -1, "dup2", err);
        GOTO_IF(dup2(fds[0], STDOUT_FILENO) == -1, "dup2", err);
        close(fds[0]);
        fds[0] = -1;

        GOTO_IF(execl(pathToChild, "child_lab3", pair1.wsemname,
                      pair1.revsemname, NULL),
                "execl", err);
    } else if (pids[1] == 0) {  // child2
        close(fds[0]);
        fds[0] = -1;

        GOTO_IF(dup2(shm2.fd, STDIN_FILENO) == -1, "dup2", err);
        GOTO_IF(dup2(fds[1], STDOUT_FILENO) == -1, "dup2", err);
        close(fds[1]);
        fds[1] = -1;

        GOTO_IF(execl(pathToChild, "child_lab3", pair2.wsemname,
                      pair2.revsemname, NULL),
                "execl", err);
    } else {  // parent
        close(fds[0]);
        fds[0] = -1;
        close(fds[1]);
        fds[1] = -1;
        while ((nread = getline(&line, &len, stream)) != -1) {
            if (nread <= FILTER_LEN) {
                GOTO_IF(SemTimedWait(pair1.wsemptr) == -1, "semTimeout", err);
                strncpy(shm1.ptr + 1, line, nread + 1);
                shm1.ptr[0] = 0;
                sem_post(pair1.revsemptr);
            } else {
                GOTO_IF(SemTimedWait(pair2.wsemptr) == -1, "semTimeout", err);
                strncpy(shm2.ptr + 1, line, nread + 1);
                shm2.ptr[0] = 0;
                sem_post(pair2.revsemptr);
            }
        }

        GOTO_IF(SemTimedWait(pair1.wsemptr) == -1, "semTimeout", err);
        GOTO_IF(SemTimedWait(pair2.wsemptr) == -1, "semTimeout", err);
        shm1.ptr[0] = -1;
        shm2.ptr[0] = -1;
        sem_post(pair2.revsemptr);
        sem_post(pair1.revsemptr);

        GOTO_IF(errno == ENOMEM, "getline", err);

        for (int i = 0; i < 2; ++i) {
            int status;
            int waitPid = wait(&status);
            GOTO_IF(status || !(waitPid == pids[0] || waitPid == pids[1]),
                    "wait", err);
        }

        free(line);
        DestroySemaphorePair(&pair1);
        DestroySemaphorePair(&pair2);
        DestroySharedMemory(&shm1);
        DestroySharedMemory(&shm2);
    }

    return 0;

err:
    free(line);
    free(extraLine);
    errno = 0;
    close(fds[0]);
    close(fds[1]);
    DestroySemaphorePair(&pair1);
    DestroySemaphorePair(&pair2);
    DestroySharedMemory(&shm1);
    DestroySharedMemory(&shm2);
    if (errno == EIO) {
        abort();
    }
    return -1;
}
\end{lstlisting}

child.c
\begin{lstlisting}[language=C++]
#include <errno.h>
#include <fcntl.h>
#include <semaphore.h>
#include <stdio.h>
#include <stdlib.h>
#include <string.h>
#include <sys/mman.h>
#include <sys/wait.h>
#include <time.h>
#include <unistd.h>

#include "error_handling.h"
#include "utils.h"

int main(int argc, char* argv[]) {
    (void)argc;
    char* ptr = (char*)mmap(NULL, MAP_SIZE, PROT_READ | PROT_WRITE, MAP_SHARED,
                            STDIN_FILENO, 0);
    SemaphorePair pair;

    GOTO_IF(CreateSemaphorePair(&pair, argv[1], argv[2]) == -1,
            "CreateSemaphorePair", err);
    while (1) {
        SemTimedWait(pair.revsemptr);
        if (ptr[0] == 0) {
            size_t len = strlen(ptr + 1);
            ReverseString(ptr + 1, len - 1);
            printf("%s", ptr + 1);
            sem_post(pair.wsemptr);
        } else {
            break;
        }
    }
    ABORT_IF(munmap(ptr, MAP_SIZE), "munmap");
    ABORT_IF(sem_close(pair.wsemptr), "sem_close");
    ABORT_IF(sem_close(pair.revsemptr), "sem_close");
    return 0;
err:
    ABORT_IF(munmap(ptr, MAP_SIZE), "munmap");
    ABORT_IF(sem_close(pair.wsemptr), "sem_close");
    ABORT_IF(sem_close(pair.revsemptr), "sem_close");
    return -1;
}
\end{lstlisting}

utils.c
\begin{lstlisting}[language=C++]
#include "utils.h"

#include <fcntl.h>
#include <semaphore.h>
#include <stdio.h>
#include <stdlib.h>
#include <sys/mman.h>
#include <time.h>
#include <unistd.h>

#include "error_handling.h"

void ReverseString(char* string, size_t length) {
    for (size_t i = 0; i < length >> 1; ++i) {
        char temp = string[i];
        string[i] = string[length - i - 1];
        string[length - i - 1] = temp;
    }
}

int CreateSharedMemory(SharedMemory* shm, const char* name, size_t size) {
    shm->fd = shm_open(name, O_CREAT | O_RDWR, S_IRUSR | S_IWUSR);
    if (shm->fd == -1) {
        return -1;
    }
    int status = ftruncate(shm->fd, (off_t)size);
    if (status == -1) {
        return -1;
    }
    shm->ptr =
        (char*)mmap(NULL, size, PROT_READ | PROT_WRITE, MAP_SHARED, shm->fd, 0);
    if (!shm->ptr) {
        return -1;
    }
    shm->name = name;
    shm->size = size;
    return 0;
}

void DestroySharedMemory(SharedMemory* shm) {
    ABORT_IF(shm_unlink(shm->name) == -1, "shm_unlink");
    ABORT_IF(munmap(shm->ptr, shm->size) == -1, "munmap");
}

int SemTimedWait(sem_t* sem) {
    struct timespec absoluteTime;
    if (clock_gettime(CLOCK_REALTIME, &absoluteTime) == -1) {
        return -1;
    }
    absoluteTime.tv_sec += 10;
    return sem_timedwait(sem, &absoluteTime);
}

int CreateSemaphorePair(SemaphorePair* pair, const char* wname,
                        const char* revname) {
    pair->wsemptr = sem_open(wname, O_CREAT, S_IRUSR | S_IWUSR, 1);
    if (!pair->wsemptr) {
        return -1;
    }
    pair->wsemname = wname;
    pair->revsemptr = sem_open(revname, O_CREAT, S_IRUSR | S_IWUSR, 0);
    if (!pair->revsemptr) {
        return -1;
    }
    pair->revsemname = revname;
    return 0;
}

void DestroySemaphorePair(SemaphorePair* pair) {
    ABORT_IF(sem_unlink(pair->wsemname) == -1, "sem_unlink");
    ABORT_IF(sem_close(pair->wsemptr) == -1, "sem_close");
    ABORT_IF(sem_unlink(pair->revsemname) == -1, "sem_unlink");
    ABORT_IF(sem_close(pair->revsemptr) == -1, "sem_close");
}
\end{lstlisting}

main.c
\begin{lstlisting}[language=C++]
#include <stdio.h>
#include <stdlib.h>

#include "lab3.h"

int main(void) {
    const char *childPath = getenv("PATH_TO_CHILD");
    if (!childPath) {
        fprintf(stderr, "Set environment variable PATH_TO_CHILD\n");
        exit(EXIT_FAILURE);
    }
    return ParentRoutine(childPath , stdin);
}
\end{lstlisting}

\newpage
\section{Тесты}
\begin{lstlisting}[language=C++]
#include <gtest/gtest.h>

extern "C" {
#include "lab3.h"
}

#include <array>
#include <filesystem>
#include <fstream>
#include <memory>

namespace fs = std::filesystem;

TEST(FirstLabTests, SimpleTest) {
    const char *childPath = getenv("PATH_TO_CHILD");
    ASSERT_TRUE(childPath);

    const std::string fileWithInput = "input.txt";
    const std::string fileWithOutput1 = "output1.txt";
    const std::string fileWithOutput2 = "output2.txt";

    constexpr int outputSize1 = 5;
    constexpr int outputSize2 = 3;
    constexpr int inputSize = outputSize1 + outputSize2;

    const std::array<std::string, inputSize> input = {
        "rev",
        "revvvvv",
        "162te16782r18223r2",
        "",
        "r",
        "pqwrpqlwfpqwoglqwpglqpgwpqw.,g;q.wg;q.wg;w.qg;.w",
        "12345678900",
        "1234567890",
    };

    const std::array<std::string, outputSize1> expectedOutput1 = {
        "ver",
        "vvvvver",
        "",
        "r",
        "0987654321",
    };

    const std::array<std::string, outputSize2> expectedOutput2 = {
        "2r32281r28761et261",
        "w.;gq.w;gw.q;gw.q;g,.wqpwgpqlgpwqlgowqpfwlqprwqp",
        "00987654321",
    };

    {
        auto inFile = std::ofstream(fileWithInput);
        inFile << fileWithOutput1 << '\n'
                << fileWithOutput2 << '\n';
        for (const auto &line : input) {
            inFile << line << '\n';
        }
    }

    auto deleter = [](FILE *file) {
        fclose(file);
    };

    std::unique_ptr<FILE, decltype(deleter)> inFile(fopen(fileWithInput.c_str(), "r"), deleter);

    ASSERT_EQ(ParentRoutine(childPath, inFile.get()), 0);

    auto outFile1 = std::ifstream(fileWithOutput1);
    auto outFile2 = std::ifstream(fileWithOutput2);
    ASSERT_TRUE(outFile1.good() && outFile2.good());

    std::string result;

    for (const std::string &line : expectedOutput1) {
        std::getline(outFile1, result);
        EXPECT_EQ(result, line);
    }

    for (const std::string &line : expectedOutput2) {
        std::getline(outFile2, result);
        EXPECT_EQ(result, line);
    }

    auto removeIfExists = [](const std::string &path) {
        if (fs::exists(path)) {
            fs::remove(path);
        }
    };

    removeIfExists(fileWithInput);
    removeIfExists(fileWithOutput1);
    removeIfExists(fileWithOutput2);
}

TEST(FirstLabTests, ZeroOutputFileTest) {
    const char *childPath = getenv("PATH_TO_CHILD");
    ASSERT_TRUE(childPath);
    const std::string fileWithInput = "input.txt";

    {
        auto inFile = std::ofstream(fileWithInput);
    }
    auto deleter = [](FILE *file) {
        fclose(file);
    };

    std::unique_ptr<FILE, decltype(deleter)> inFile(fopen(fileWithInput.c_str(), "r"), deleter);

    ASSERT_EQ(ParentRoutine(childPath, inFile.get()), -1);
}

TEST(FirstLabTests, OneOutputFileTest) {
    const char *childPath = getenv("PATH_TO_CHILD");
    ASSERT_TRUE(childPath);
    const std::string fileWithInput = "input.txt";
    const std::string fileWithOutput = "output.txt";

    {
        auto inFile = std::ofstream(fileWithInput);
        inFile << fileWithOutput << '\n';
    }
    auto deleter = [](FILE *file) {
        fclose(file);
    };

    std::unique_ptr<FILE, decltype(deleter)> inFile(fopen(fileWithInput.c_str(), "r"), deleter);

    ASSERT_EQ(ParentRoutine(childPath, inFile.get()), -1);
    ASSERT_FALSE(fs::exists(fileWithOutput));
}

TEST(FirstLabTests, EmptyInputTest) {
    const char *childPath = getenv("PATH_TO_CHILD");
    ASSERT_TRUE(childPath);

    const std::string fileWithInput = "input.txt";
    const std::string fileWithOutput1 = "output1.txt";
    const std::string fileWithOutput2 = "output2.txt";

    {
        auto inFile = std::ofstream(fileWithInput);
        inFile << fileWithOutput1 << '\n'
                << fileWithOutput2 << '\n';
    }

    auto deleter = [](FILE *file) {
        fclose(file);
    };

    std::unique_ptr<FILE, decltype(deleter)> inFile(fopen(fileWithInput.c_str(), "r"), deleter);

    ASSERT_EQ(ParentRoutine(childPath, inFile.get()), 0);

    auto outFile1 = std::ifstream(fileWithOutput1);
    auto outFile2 = std::ifstream(fileWithOutput2);
    ASSERT_TRUE(outFile1.good() && outFile2.good());
    ASSERT_TRUE(fs::is_empty(fileWithOutput1) && fs::is_empty(fileWithOutput2));

    auto removeIfExists = [](const std::string &path) {
        if (fs::exists(path)) {
            fs::remove(path);
        }
    };

    removeIfExists(fileWithInput);
    removeIfExists(fileWithOutput1);
    removeIfExists(fileWithOutput2);
}
\end{lstlisting}
\newpage

\section{Консоль}
\begin{verbatim}
ivan@asus-vivobook ~/c/o/b/lab3 (reports)> PATH_TO_CHILD=child_lab3 lab3
1
2
sds
dfsdfs
sdf
s
ccvvf
df

fdffffffffffffffg2f52g2
ivan@asus-vivobook ~/c/o/b/lab3 (reports)> ls
1  2  CMakeFiles/  CTestTestfile.cmake  Makefile  child_lab3*  cmake_install.cmake  input  lab3*
ivan@asus-vivobook ~/c/o/b/lab3 (reports)> cat 1 
sds
sfdsfd
fds
s
fvvcc
fd

ivan@asus-vivobook ~/c/o/b/lab3 (reports)> cat 2
2g25f2gffffffffffffffdf
\end{verbatim}

\section{Запуск тестов}
\begin{verbatim}
ivan@asus-vivobook ~/c/o/b/tests (reports) [SIGINT]> PATH_TO_CHILD=../lab3/child_lab3 ./lab3_test
Running main() from /var/tmp/portage/dev-cpp/gtest-1.13.0/work/googletest-1.13.0/googletest/src/gtest_main.cc
[==========] Running 4 tests from 1 test suite.
[----------] Global test environment set-up.
[----------] 4 tests from FirstLabTests
[ RUN      ] FirstLabTests.SimpleTest
[       OK ] FirstLabTests.SimpleTest (9 ms)
[ RUN      ] FirstLabTests.ZeroOutputFileTest
Input 2 filenames
[       OK ] FirstLabTests.ZeroOutputFileTest (0 ms)
[ RUN      ] FirstLabTests.OneOutputFileTest
Input 2 filenames
[       OK ] FirstLabTests.OneOutputFileTest (0 ms)
[ RUN      ] FirstLabTests.EmptyInputTest
[       OK ] FirstLabTests.EmptyInputTest (4 ms)
[----------] 4 tests from FirstLabTests (15 ms total)

[----------] Global test environment tear-down
[==========] 4 tests from 1 test suite ran. (16 ms total)
[  PASSED  ] 4 tests.
\end{verbatim}
\newpage

\section{Выводы}

В результате выполнения данной лабораторной работы была написана программа на языке С, осуществляющая работу с процессами и взаимодействие между ними через системные сигналы и отображаемые файлы. Приобретены практические навыки в освоении принципов работы с файловыми системами и обеспечении обмена данных между процессами посредством технологии «File mapping».
\end{document}