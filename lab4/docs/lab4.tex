\documentclass[a4paper, 12pt]{article}
\usepackage{cmap}
\usepackage[12pt]{extsizes}			
\usepackage{mathtext} 				
\usepackage[T2A]{fontenc}			
\usepackage[utf8]{inputenc}			
\usepackage[english,russian]{babel}
\usepackage{setspace}
\singlespacing
\usepackage{amsmath,amsfonts,amssymb,amsthm,mathtools}
\usepackage{fancyhdr}
\usepackage{soulutf8}
\usepackage{euscript}
\usepackage{mathrsfs}
\usepackage{listings}

\usepackage[colorlinks=true, urlcolor=blue, linkcolor=black]{hyperref}

\pagestyle{fancy}
\usepackage{indentfirst}
\usepackage[top=10mm]{geometry}
\rhead{}
\lhead{}
\renewcommand{\headrulewidth}{0mm}
\usepackage{tocloft}
\renewcommand{\cftsecleader}{\cftdotfill{\cftdotsep}}
\usepackage[dvipsnames]{xcolor}

\lstdefinestyle{mystyle}{ 
	keywordstyle=\color{OliveGreen},
	numberstyle=\tiny\color{Gray},
	stringstyle=\color{BurntOrange},
	basicstyle=\ttfamily\footnotesize,
	breakatwhitespace=false,         
	breaklines=true,                 
	captionpos=b,                    
	keepspaces=true,                 
	numbers=left,                    
	numbersep=5pt,                  
	showspaces=false,                
	showstringspaces=false,
	showtabs=false,                  
	tabsize=2
}

\lstset{style=mystyle}

\begin{document}
\thispagestyle{empty}
\begin{center}
    Московский авиационный институт

    (Национальный исследовательский университет)

    Факультет "Информационные технологии и прикладная математика"

    Кафедра "Вычислительная математика и программирование"

\end{center}
\vspace{40ex}
\begin{center}
    \textbf{\large{Лабораторная работа №4 по курсу\linebreak \textquotedblleft Операционные системы\textquotedblright}}
\end{center}
\vspace{35ex}
\begin{flushright}
    \textit{Студент: } Кочкожаров Иван Вячеславович

    \vspace{2ex}
    \textit{Группа: } М8О-208Б-22

    \vspace{2ex}
    \textit{Преподаватель: } Миронов Евгений Сергеевич

    \vspace{2ex}
    \textit{Вариант: } 25

    \vspace{2ex}
    \textit{Оценка: } \underline{\quad\quad\quad\quad\quad\quad}

    \vspace{2ex}
    \textit{Дата: } \underline{\quad\quad\quad\quad\quad\quad}

    \vspace{2ex}
    \textit{Подпись: } \underline{\quad\quad\quad\quad\quad\quad}

\end{flushright}

\vspace{5ex}

\begin{vfill}
    \begin{center}
        Москва, 2023
    \end{center}
\end{vfill}
\newpage

\begingroup
\color{black}
\tableofcontents\newpage
\endgroup

\section{Репозиторий}
\href{https://github.com/kochkozharov/os-labs}{https://github.com/kochkozharov/os-labs}

\section{Цель работы}
Приобретение практических навыков в:
\begin{itemize}
    \item Создании динамических библиотек
    \item Создании программ, которые используют функции динамических библиотек
\end{itemize}

\section{Задание}
Требуется создать динамические библиотеки, которые реализуют определенный функционал. Далее использовать данные библиотеки 2-мя способами:
\begin{enumerate}
    \item Во время компиляции (на этапе «линковки»/linking)
    \item Во время исполнения программы. Библиотеки загружаются в память с помощью интерфейса ОС для работы с динамическими библиотеками
\end{enumerate}
В конечном итоге, в лабораторной работе необходимо получить следующие части:
\begin{itemize}
    \item Динамические библиотеки, реализующие контракты, которые заданы вариантом;
    \item Тестовая программа (программа №1), которая используют одну из библиотек, используя знания полученные на этапе компиляции;
    \item Тестовая программа (программа №2), которая загружает библиотеки, используя только их местоположение и контракты.
\end{itemize}

\section{Описание работы программы}
Функции, написанные в результате выполнения лабораторной работы:
\begin{itemize}
    \item Подсчёт наибольшего общего делителя для двух натуральных чисел
    \item Перевод числа x из десятичной системы счисления в другую
\end{itemize}

В ходе выполнения лабораторной работы я использовала следующие системные вызовы:
\begin{itemize}
    \item dlopen - открытие динамического объекта
    \item dlclose - закрытие динамического объекта
    \item dlsym - получить адрес, по которому символ из библиотеки расположен в памяти
\end{itemize}


\newpage

lib1.c
\section{Исходный код}
\begin{lstlisting}[language=C++]
#include <stdlib.h>
#include <string.h>

#include "lib.h"
#include "utils.h"

char* Translation(long x) {
    if (x < 0) {
        return nullptr;
    }
    char* binary = (char*)malloc(NUM_BUFFER_SIZE);
    if (!binary) {
        return binary;
    }
    int i = 0;
    do {
        binary[i++] = '0' + (x & 1);
        x >>= 1;
    } while(x);
    binary[i] = '\0';
    ReverseString(binary, i);
    return binary;
}

int GCD(int a, int b) {
    if (a < 0 || b < 0) {
        return -1;
    }
    while (b != 0) {
        int tmp;
        tmp = a % b;
        a = b;
        b = tmp;
    }
    return a;
}
\end{lstlisting}

lib2.cpp
\begin{lstlisting}[language=C++]
#include <stdlib.h>
#include <string.h>

#include "lib.h"
#include "utils.h"

char* Translation(long x) {
    if (x < 0) {
        return nullptr;
    }
    char* trinary = (char*)malloc(NUM_BUFFER_SIZE);
    if (!trinary) {
        return trinary;
    }
    int i = 0;
    do {
        trinary[i++] = '0' + (x % 3);
        x /= 3;
    } while (x);
    trinary[i] = '\0';
    ReverseString(trinary, i);
    return trinary;
}

int GCD(int a, int b) {
    if (a < 0 || b < 0) {
        return -1;
    }
    int min = a < b ? a : b;
    int gcd = 0;
    for (int i = 1; i < min + 1; i++) {
        if ((a % i) == 0 && (b % i) == 0) {
            gcd = i;
        }
    }
    return gcd;
}
\end{lstlisting}

utils.cpp
\begin{lstlisting}[language=C++]
#include "utils.h"
#include <cstddef>

void ReverseString(char* string, std::size_t length) {
    for (std::size_t i = 0; i < length >> 1; ++i) {
        char temp = string[i];
        string[i] = string[length - i - 1];
        string[length - i - 1] = temp;
    }
}
\end{lstlisting}


\newpage
\section{Тесты}
lab4\_lib1\_test.cpp
\begin{lstlisting}[language=C++]
#include <gtest/gtest.h>

#include <cstring>

#include "lib.h"

TEST(FourthLabTests, GCDTest) {
    EXPECT_EQ(GCD(49, 56), 7);
    EXPECT_EQ(GCD(56, 49), 7);
    EXPECT_EQ(GCD(57, 49), 1);
}

TEST(FourthLabTests, TranslationTest) {
    auto deleter = [](char *str) { std::free(str); };
    std::unique_ptr<char, decltype(deleter)> str(Translation(31), deleter);
    EXPECT_TRUE(std::strcmp(str.get(), "11111") == 0);
    str.reset(Translation(0));
    EXPECT_TRUE(std::strcmp(str.get(), "0") == 0);
    str.reset(Translation(888));
    EXPECT_TRUE(std::strcmp(str.get(), "1101111000") == 0);
    str.reset(Translation(-1));
    EXPECT_EQ(str.get(), nullptr);
}
\end{lstlisting}
lab4\_lib2\_test.cpp
\begin{lstlisting}[language=C++]
#include <gtest/gtest.h>

#include <cstring>
#include <utility>

#include "lib.h"

TEST(FourthLabTests, GCDTest) {
    EXPECT_EQ(GCD(49, 56), 7);
    EXPECT_EQ(GCD(56, 49), 7);
    EXPECT_EQ(GCD(57, 49), 1);
}

TEST(FourthLabTests, TranslationTest) {
    auto deleter = [](char *str) { std::free(str); };
    std::unique_ptr<char, decltype(deleter)> str(Translation(0), deleter);
    EXPECT_TRUE(std::strcmp(str.get(), "0") == 0);
    str.reset(Translation(5));
    EXPECT_TRUE(std::strcmp(str.get(), "12") == 0);
    str.reset(Translation(888));
    EXPECT_TRUE(std::strcmp(str.get(), "1012220") == 0);
    str.reset(Translation(-1));
    EXPECT_EQ(str.get(), nullptr);
}
\end{lstlisting}
\newpage

\section{Консоль}
\begin{verbatim}
ivan@asus-vivobook ~/c/o/b/lab4 (reports)> static_main
> 1 8 81
1
> 2 300
100101100
> exit
ivan@asus-vivobook ~/c/o/b/lab4 (reports)> cat run.sh
export PATH_TO_LIB1="./liblib1.so"
export PATH_TO_LIB2="./liblib2.so"
./dynamic_main
ivan@asus-vivobook ~/c/o/b/lab4 (reports)> run.sh
> 1 81 9
9
> 2 443
110111011
> 0
Switched
> 1 81 9
9
> 2 443
121102
\end{verbatim}

\section{Запуск тестов}
\begin{verbatim}
ivan@asus-vivobook ~/c/o/b/tests (reports)> lab4_lib1_test
Running main() from /var/tmp/portage/dev-cpp/gtest-1.13.0/work/googletest-1.13.0/googletest/src/gtest_main.cc
[==========] Running 2 tests from 1 test suite.
[----------] Global test environment set-up.
[----------] 2 tests from FourthLabTests
[ RUN      ] FourthLabTests.GCDTest
[       OK ] FourthLabTests.GCDTest (0 ms)
[ RUN      ] FourthLabTests.TranslationTest
[       OK ] FourthLabTests.TranslationTest (0 ms)
[----------] 2 tests from FourthLabTests (0 ms total)

[----------] Global test environment tear-down
[==========] 2 tests from 1 test suite ran. (0 ms total)
[  PASSED  ] 2 tests.
ivan@asus-vivobook ~/c/o/b/tests (reports)> lab4_lib2_test
Running main() from /var/tmp/portage/dev-cpp/gtest-1.13.0/work/googletest-1.13.0/googletest/src/gtest_main.cc
[==========] Running 2 tests from 1 test suite.
[----------] Global test environment set-up.
[----------] 2 tests from FourthLabTests
[ RUN      ] FourthLabTests.GCDTest
[       OK ] FourthLabTests.GCDTest (0 ms)
[ RUN      ] FourthLabTests.TranslationTest
[       OK ] FourthLabTests.TranslationTest (0 ms)
[----------] 2 tests from FourthLabTests (0 ms total)

[----------] Global test environment tear-down
[==========] 2 tests from 1 test suite ran. (0 ms total)
[  PASSED  ] 2 tests.
\end{verbatim}
\newpage

\section{Выводы}

В результате выполнения данной лабораторной работы были созданы динамические библиотеки, которые реализуют функционал в соответствие с вариантом задания на С++. Приобретены практические навыки в создании программ, которые используют функции динамических библиотек.
\end{document}